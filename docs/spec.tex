% !TEX encoding = UTF-8 Unicode

\documentclass[12pt]{article}
\usepackage[utf8]{inputenc}
\usepackage[T1]{fontenc}
\usepackage[czech]{babel}
\usepackage[left=2cm, right=2cm, top=3cm, bottom=3cm]{geometry}
\usepackage[fleqn]{mathtools}
\usepackage{amsfonts,amssymb}
\usepackage{enumitem}
\usepackage{icomma}
\usepackage{tikz}
\usepackage{etoolbox}
\usepackage{forest}
\usepackage{amsthm}
\usepackage[hidelinks]{hyperref}
\usepackage{url}
\usepackage{relsize,xspace}

% Hlavička.
\usepackage{fancyhdr}
\pagestyle{fancy}
\rhead{Jan Joneš}
\lhead{}
\renewcommand{\headrulewidth}{0pt}

% Styl URL.
\DeclareUrlCommand\url{\urlstyle{tt}}

% C++.
\newcommand{\Rplus}{\protect\hspace{-.1em}\protect\raisebox{.35ex}{\smaller{\smaller\textbf{+}}}}
\newcommand{\Cpp}{\mbox{C\Rplus\Rplus}\xspace}

% IEEE citace.
\makeatletter
\def\bstctlcite{\@ifnextchar[{\@bstctlcite}{\@bstctlcite[@auxout]}}
\def\@bstctlcite[#1]#2{\@bsphack
  \@for\@citeb:=#2\do{%
    \edef\@citeb{\expandafter\@firstofone\@citeb}%
    \if@filesw\immediate\write\csname #1\endcsname{\string\citation{\@citeb}}\fi}%
  \@esphack}
\makeatother

\begin{document}

\bstctlcite{BSTcontrol}

\section*{Specifikace programu ipaSim}

\subsection*{Úvod}

ipaSim bude program umožňující spouštět aplikace napsané pro operační systém iOS na zařízeních s operačním systémem Windows.

Vstupem bude balíček *.ipa, což je standardní formát pro distribuci aplikací na iOS a macOS.
Uvnitř balíčku je mimo jiné také strojový kód aplikace, většinou pro architekturu ARM, na které operační systém iOS běží.
Tento strojový kód bude ipaSim emulovat a funkce iOS, které bude emulovaná aplikace volat, se přeloží na odpovídající funkce Windows.

Díky tomu se bude emulovaná aplikace chovat jako nativní aplikace napsaná pro Windows.
Zároveň nebude nutné emulovat žádnou část systému iOS, pouze kód aplikace.

\subsection*{Podobné projekty}

Wine~\cite{wine} je linuxový nástroj, který umožňuje spouštět Windows aplikace.
K tomu využívá převod mezi Win32 API (tedy funkcemi operačního systému Windows) a funkcemi Linuxu, podobně jako to bude dělat naše aplikace.

Darling~\cite{darling} je velice podobný program, který umožňuje spouštění macOS aplikací na Linuxu.
Tento projekt je ovšem ve velmi rané fázi a umí emulovat pouze jednoduché konzolové aplikace \cite{darlingStatus}.

\subsection*{Platforma}

ipaSim bude UWP\footnote{\url{https://docs.microsoft.com/en-us/windows/uwp/}} aplikace napsaná v jazyce \Cpp pomocí nadstavby \Cpp/WinRT\footnote{\url{https://docs.microsoft.com/en-us/windows/uwp/cpp-and-winrt-apis/}}.
To je sada hlavičkových souborů obalujících WinRT API, tedy funkce Windows dostupné na všech zařízeních s Windows 10 (desktopech, mobilních zařízeních a dalších).
Jazyk \Cpp je vhodný, protože knihovny, které budeme používat, v podstatě nemají obdoby v jiných programovacích jazycích.
Zároveň bude ipaSim velice nízkoúrovňová aplikace pracující často přímo s pamětí, na což je také jazyk \Cpp ideální.
Založit projekt na platformě UWP je nutné, protože stěžejní knihovna WinObjC (viz sekci~\nameref{sec:winobjc}) je na ní založená.

\subsection*{Načítání balíčků}

Při přeložení aplikace napsané pro iOS (například pomocí IDE Xcode\footnote{\url{https://developer.apple.com/xcode/}}) vznikne balíček obsahující všechny potřebné soubory k běhu aplikace na cílovém zařízení.
Tento balíček má většinou příponu \texttt{*.ipa} (zkratka z \textit{iPhone application}), je to normální archiv ZIP a obsahuje metadata, multimediální zdroje (ikony, obrázky apod.) a samotný spustitelný soubor \cite{ipa}.
Nás bude zajímat hlavně posledně zmíněná část -- spustitelný soubor, který ipaSim musí načíst do paměti pro emulování.
Ten má formát Mach-O, případně Fat Mach-O, který obsahuje více Mach-O částí pro různé architektury (např. ARM a ARM64) \cite{macho}.

Mach-O formát je podobný jiným formátům spustitelných souborů (např. PE a ELF).
Obsahuje hlavičku popisující jeho strukturu, tabulku symbolů, použité sdílené knihovny a v neposlední řadě i segmenty a sekce obsahující kód a data.
My na jeho čtení můžeme použít například knihovnu LIEF\footnote{\url{https://github.com/lief-project/LIEF}}.

Pro emulaci aplikací budeme napodobovat chování systému iOS při načítání aplikací do paměti \cite{execMacho}.
Nejprve jednoduše načteme všechny sekce obsažené v Mach-O souboru do operační paměti.
Poté provedeme relokace a dynamické linkování s knihovnami operačního systému.
Zde se právě provede převod mezi funkcemi iOS a Windows -- emulovaná aplikace totiž samozřejmě závisí na funkcích iOS, my ji ale slinkujeme s funkcemi WinObjC (viz sekci~\nameref{sec:winobjc}).

\subsection*{Emulování kódu}

Pro emulování kódu použijeme knihovnu Unicorn, která je založená na emulátoru Qemu (na rozdíl od něj ale nevyžaduje emulování celého operačního systému).
Jemu prostě řekneme, kde je v paměti načtený kód a odkud jej má začít provádět.
Pokud aplikace zavolá funkci operačního systému, vyskočí ven z namapované paměti, což zachytíme a volání převedeme na volání OS Windows.

\subsection*{Volání OS}
\label{sec:winobjc}

Pro převádění mezi voláními iOS a Windows použijeme knihovnu WinObjC.
Ta poskytuje API funkce přesně stejné jako systémové knihovny iOS, a tyto funkce jsou implementované pomocí volání tzv. WinRT API, tedy moderního API OS Windows (dostupného od Windows 10).
Malý zádrhel je pouze v tom, že tato knihovna je navržená pro fungování se zdrojovými kódy iOS aplikací, kdežto mi chceme emulovat již přeložené spustitelné soubory.
Problém spočívá v použití runtime jazyka Objective-C, který není binárně kompatibilní s tím, který používá Apple v systému iOS.
To není problém pro WinObjC, kde se zdrojové kódy prostě přeloží pro ten správný runtime, ale je to problém pro naši aplikaci, kde jsou kódy už přeložené pro konkrétní a nekompatibilní runtime.

\subsection*{Běhové prostředí jazyka Objective-C}

Aplikace pro iOS jsou napsané v jazyce Objective-C, který je velmi dynamický a pro běh programů v něm napsaných se spoléhá na svojí runtime knihovnu, jejíž funkce přeložené programy volají.
Tyto funkce zahrnují například funkce pro posílání zpráv mezi objekty, což je základní forma komunikace mezi objekty v Objective-C (nahrazuje volání funkcí), nebo funkce pro dynamickou introspekci objektů.
Zároveň se runtime spoléhá na datové struktury vygenerované překladačem, které jsou uloženy v Mach-O souboru.
Ty obsahují například informace o třídách a metodách použitých v programu.
A právě tyto datové struktury nejsou mezi jednotlivými runtime knihovnami kompatibilní.
Proto je nutné použít runtime, který bude tyto struktury interpretovat správně, tak jak by to dělal runtime v iOS.

Pro Windows existují různé implementace Objective-C (\textbf{TODO:} příklady), žádná ale není kompatibilní s runtime knihovnou od Applu.
Proto bude naše aplikace obsahovat vlastní Objective-C runtime, který bude pravděpodobně pouze port runtime knihovny od Applu.
Ta je napsaná v \Cpp, a tak můžeme přenositelné části zachovat a přepsat pouze kód závislý na OS.
Využijeme také, že tato knihovna závisí na | POSIXových vláknech, pro které existuje na Windows port winpthreads.

\bibliographystyle{IEEEtran}
\bibliography{spec}

\end{document}
